%%%%%%%%%%%%%%%%%%%%%%%%%%%%%%%%%%%%%%%%
%%%%%%%% INTRODUCTION file
%%%%%%%%%%%%%%%%%%%%%%%%%%%%%%%%%%%%%%%%
 
\privatize
\message{INTRODUCTION}
 
This document describes a \TeX\ macro package developed within the 
Scientific
and Technical Information (STI) Department of SACLANTCEN for preparing
scientific documents for publication (included in this category are 
Management
documents which relate to the scientific programme). 
\medskip
A fairly extensive set of macros specifically tailored for SACLANTCEN 
documents
has been layered on top of the {\tt PLAIN} base.  This format has been
preloaded into \TeX\ and is called \cen.  The aims of \cen\ are to provide
appropriate formatting for different document types, to facilitate some of 
the
more complex document structures (such as tabular information, page
composition, etc.), to enable a simple means to integrate graphics into the
document, to encourage a certain discipline in file-naming, presentation, 
etc.,
and to elicit and verify document declarations. 

\subhead
File Organization                                                   

The \cen\ format is distributed over a set of component macro files 
located in
a dedicated directory denoted by the logical `{\tt CENTEX:}'. The annotated
source files (of type `{\tt .DOC}') were used to generate this draft. By
invoking the {\tt EX\TeX} utility program through the following command, 
the
files can be reduced to their  daughter `{\tt .TEX}' versions:
$$
\vbox{\listing
\$ CEN/BUILD=EXE CENTEX.DOC
\endlisting}
$$
Both `{\tt .DOC}' files and `{\tt .TEX}' files are booted, either as \TeX\ 
data
or as \TeX\ procedures, from the master file `{\tt CENTEX}'. This works by
maintaining the set of component files in the list macro \.\cen@module@set.
which is activated when the `{\tt CENTEX}' file is read by {\tt INITEX}. 
Note
that for the `{\tt .DOC}' files \.\cen@module@set. is redefined at the end 
of
the `{\tt CENTEX.DOC}' file in case any alterations have been made to the 
list
of component files. 
\medskip
In order to keep the format lighter only those `{\tt .TEX}' files of 
general
use are preloaded.  We can thus distinguish between internal files and 
external
files (such as the Programme of Work macro file) which are needed either
irregularly or by particular users only. In this regard we can make mention
of the change file `{\tt CENTEX.CH}' which is loaded each time that \cen\ 
is invoked
by a member of the STI group. This file can be used to try out new routines
in a controlled fashion without compromising the system for other users.
When a particular change or set of changes is felt to be sufficiently 
robust
a new version of \cen\ can be prepared.

\subhead
Invoking \cenTeX\

The \cen\ format makes use of two auxiliary files located in the user's 
default
directory: `{\tt DATA\-FILE.STI}' containing user-supplied document
declarations, and `{\tt CEN\$DATA.STI}' containing dynamic (runtime) 
system-supplied declarations. A front-end program `{\tt CEN}'  generates 
the
`{\tt CEN\$DATA}' file, calls \cen\ and regulates the post-processing 
options.
This is described more fully in the \cendoc\  system manual (I), which also
shows the form of the auxiliary files.

\subhead
About This Document

This document was generated by simply issuing the following
command at the DCL prompt:
$$
\vbox{\listing
\$ CEN/BUILD=DOC CENTEX.DOC
\endlisting}
$$
(Index entries are written out onto the file `{\tt INDEX.INX}' which is 
case-insensitive sorted and massaged by the {\tt SOR\TeX} utility 
program.) 
 
\noprivatize
 
%%%%%%%%%%%%%%%%%%%%%%%%%%%%%%%%%%%%%%%%
